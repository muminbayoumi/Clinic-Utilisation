\PassOptionsToPackage{unicode=true}{hyperref} % options for packages loaded elsewhere
\PassOptionsToPackage{hyphens}{url}
%
\documentclass[]{article}
\usepackage{lmodern}
\usepackage{amssymb,amsmath}
\usepackage{ifxetex,ifluatex}
\usepackage{fixltx2e} % provides \textsubscript
\ifnum 0\ifxetex 1\fi\ifluatex 1\fi=0 % if pdftex
  \usepackage[T1]{fontenc}
  \usepackage[utf8]{inputenc}
  \usepackage{textcomp} % provides euro and other symbols
\else % if luatex or xelatex
  \usepackage{unicode-math}
  \defaultfontfeatures{Ligatures=TeX,Scale=MatchLowercase}
\fi
% use upquote if available, for straight quotes in verbatim environments
\IfFileExists{upquote.sty}{\usepackage{upquote}}{}
% use microtype if available
\IfFileExists{microtype.sty}{%
\usepackage[]{microtype}
\UseMicrotypeSet[protrusion]{basicmath} % disable protrusion for tt fonts
}{}
\IfFileExists{parskip.sty}{%
\usepackage{parskip}
}{% else
\setlength{\parindent}{0pt}
\setlength{\parskip}{6pt plus 2pt minus 1pt}
}
\usepackage{hyperref}
\hypersetup{
            pdftitle={Clinic Utilisation Report},
            pdfauthor={M.Mohamed;M.Ali;M.Rao},
            pdfborder={0 0 0},
            breaklinks=true}
\urlstyle{same}  % don't use monospace font for urls
\usepackage[margin=1in]{geometry}
\usepackage{longtable,booktabs}
% Fix footnotes in tables (requires footnote package)
\IfFileExists{footnote.sty}{\usepackage{footnote}\makesavenoteenv{longtable}}{}
\usepackage{graphicx,grffile}
\makeatletter
\def\maxwidth{\ifdim\Gin@nat@width>\linewidth\linewidth\else\Gin@nat@width\fi}
\def\maxheight{\ifdim\Gin@nat@height>\textheight\textheight\else\Gin@nat@height\fi}
\makeatother
% Scale images if necessary, so that they will not overflow the page
% margins by default, and it is still possible to overwrite the defaults
% using explicit options in \includegraphics[width, height, ...]{}
\setkeys{Gin}{width=\maxwidth,height=\maxheight,keepaspectratio}
\setlength{\emergencystretch}{3em}  % prevent overfull lines
\providecommand{\tightlist}{%
  \setlength{\itemsep}{0pt}\setlength{\parskip}{0pt}}
\setcounter{secnumdepth}{0}
% Redefines (sub)paragraphs to behave more like sections
\ifx\paragraph\undefined\else
\let\oldparagraph\paragraph
\renewcommand{\paragraph}[1]{\oldparagraph{#1}\mbox{}}
\fi
\ifx\subparagraph\undefined\else
\let\oldsubparagraph\subparagraph
\renewcommand{\subparagraph}[1]{\oldsubparagraph{#1}\mbox{}}
\fi

% set default figure placement to htbp
\makeatletter
\def\fps@figure{htbp}
\makeatother

\usepackage{booktabs}
\usepackage{longtable}
\usepackage{array}
\usepackage{multirow}
\usepackage{wrapfig}
\usepackage{float}
\floatplacement{figure}{H}
\usepackage{colortbl}
\usepackage{pdflscape}
\usepackage{tabu}
\usepackage{threeparttable}
\usepackage{threeparttablex}
\usepackage[normalem]{ulem}
\usepackage{makecell}
\usepackage{xcolor}
\usepackage{flafter}
\usepackage{flafter}

\title{Clinic Utilisation Report}
\author{M.Mohamed;M.Ali;M.Rao}
\date{}

\begin{document}
\maketitle

{
\setcounter{tocdepth}{2}
\tableofcontents
}
\hypertarget{a-comparison-across-clinics-and-between-pilgrim-and-peripheral-sites}{%
\subsection{A comparison across clinics and between pilgrim and
peripheral
sites}\label{a-comparison-across-clinics-and-between-pilgrim-and-peripheral-sites}}

\begin{center}\rule{0.5\linewidth}{0.5pt}\end{center}

The aim of this project is to audit our use of general surgery and
colorectal surgery clinics. We acquired our clinic attendance data from
hospital information services. We further analysed this data to assess
our utilization and DNAs. These are the clinic codes use for the purpose
of this analysis \textbf{JH-MIRO2, JH-ZAIO4, PH-ATE35, PH-ATEPF,
PH-GOR52, PH-MIR41, PH-MIRPF, PH-MMC35, PH-MOB21, PH-MOB40, PH-RTH11,
PH-RTH45, PH-ZAI50, SD-MMCSD, SD-ATESD}

\begin{center}\rule{0.5\linewidth}{0.5pt}\end{center}

\newpage

\hypertarget{the-data}{%
\subsection{1.The Data}\label{the-data}}

With a preliminary view we can see that our \textcolor{blue}{Pilgrim
Median Booking Rate}:91.7\% is marginally higher than our
\textcolor{red}{Peripheral Median Booking Rate}:89.5\%. Difference in
median was found to be statistically significant at p-value of
\emph{0.023}(Graph 1.1A). Similarly our \textcolor{blue}{Pilgrim Median
Utilizaton Rate}:83.3\% is marginally higher than our
\textcolor{red}{Peripheral Median Utilization Rate}:79.2\%. Difference
in median was found to be statistically significant at p-value of
\emph{0.011}(Graph 1.1B)

\hypertarget{graph-1.1}{%
\subsubsection{Graph 1.1}\label{graph-1.1}}

\begin{center}\includegraphics{LF2_files/figure-latex/unnamed-chunk-3-1} \end{center}

\begin{table}
\caption{\label{tab:unnamed-chunk-4}Total Number of One and Two Man Clinics for Pilgrim and Peripheral Sites}

\centering
\begin{tabular}[t]{llrl}
\toprule
M & OneVsTwo & count & Site\\
\midrule
Aug & One Man & 4 & Peripheral\\
Aug & One Man & 18 & Pilgrim\\
Sep & One Man & 5 & Peripheral\\
Sep & One Man & 13 & Pilgrim\\
Oct & One Man & 5 & Peripheral\\
\addlinespace
Oct & One Man & 21 & Pilgrim\\
Nov & One Man & 6 & Peripheral\\
Nov & One Man & 22 & Pilgrim\\
Dec & One Man & 7 & Peripheral\\
Dec & One Man & 14 & Pilgrim\\
\addlinespace
Jan & One Man & 4 & Peripheral\\
Jan & One Man & 18 & Pilgrim\\
\bottomrule
\end{tabular}
\centering
\begin{tabular}[t]{llrl}
\toprule
M & OneVsTwo & count & Site\\
\midrule
Aug & Two Man & 9 & Pilgrim\\
Sep & Two Man & 9 & Pilgrim\\
Oct & Two Man & 13 & Pilgrim\\
Nov & Two Man & 8 & Pilgrim\\
Dec & Two Man & 9 & Pilgrim\\
\addlinespace
Jan & Two Man & 11 & Pilgrim\\
\bottomrule
\end{tabular}
\end{table}

\hypertarget{graph-1.2-histogram-demonstrating-the-distribution-of-clinic-utilizatation-rates}{%
\subsubsection{Graph-1.2 Histogram demonstrating the distribution of
Clinic Utilizatation
Rates}\label{graph-1.2-histogram-demonstrating-the-distribution-of-clinic-utilizatation-rates}}

\begin{center}\includegraphics{LF2_files/figure-latex/unnamed-chunk-5-1} \end{center}

\hypertarget{further-breakdown}{%
\subsection{2.Further Breakdown}\label{further-breakdown}}

The following graphs demonstrate per clinic data.
\textcolor{black}{\textbf{Black Shapes}} demonstrate booking rate while
\textcolor{red}{\textbf{Red Shapes}} demonstrate utilization rates.
These are monthly rates ie the actual figure is an average of clinics
used per month. \textbf{Booking rate} is
\[\frac{initial\ booked slots}{total\ available\ slots}\] while
\textcolor{red}{\textbf{Utilization Rate}} is
\[\frac{attended \ clinic\  slots}{booked\ slots}\]

\hypertarget{graph-2.1-booking-and-utilization-rate-per-month-for-one-man-clinics}{%
\subsubsection{Graph-2.1 Booking and Utilization rate per month for One
Man
clinics}\label{graph-2.1-booking-and-utilization-rate-per-month-for-one-man-clinics}}

\begin{center}\includegraphics{LF2_files/figure-latex/unnamed-chunk-6-1} \end{center}

\hypertarget{graph-2.2-booking-and-utilization-rates-per-month-for-two-man-clinics}{%
\subsubsection{Graph-2.2 Booking and Utilization Rates per month for Two
Man
clinics}\label{graph-2.2-booking-and-utilization-rates-per-month-for-two-man-clinics}}

\begin{center}\includegraphics{LF2_files/figure-latex/unnamed-chunk-7-1} \end{center}

\hypertarget{graph-2.3-utilization-and-booking-rate-per-month-across-peripheral-vs-pilgrim-clinics}{%
\subsubsection{Graph-2.3 Utilization and Booking Rate per month across
Peripheral vs Pilgrim
clinics}\label{graph-2.3-utilization-and-booking-rate-per-month-across-peripheral-vs-pilgrim-clinics}}

\begin{center}\includegraphics{LF2_files/figure-latex/unnamed-chunk-8-1} \end{center}

\hypertarget{graph-2.4-utilization-and-booking-rate-per-month-across-peripheral-vs-pilgrim-clinics}{%
\subsubsection{Graph-2.4 Utilization and Booking Rate per month across
Peripheral vs Pilgrim
clinics}\label{graph-2.4-utilization-and-booking-rate-per-month-across-peripheral-vs-pilgrim-clinics}}

\begin{center}\includegraphics{LF2_files/figure-latex/unnamed-chunk-9-1} \end{center}

\hypertarget{graph-2.5-per-clinic-dna-rate}{%
\subsubsection{Graph-2.5 Per Clinic DNA
Rate}\label{graph-2.5-per-clinic-dna-rate}}

\begin{center}\includegraphics{LF2_files/figure-latex/unnamed-chunk-10-1} \end{center}

\begin{center}\rule{0.5\linewidth}{0.5pt}\end{center}

\hypertarget{discussion}{%
\subsection{3.Discussion}\label{discussion}}

\protect\hyperlink{graph2.1}{Graph-2.1} and
\protect\hyperlink{graph2.2}{Graph-2.2} show a clear underbooking with
even clearer non-attendance. \protect\hyperlink{graph2.4}{Graph-2.4} \&
\protect\hyperlink{graph2.3}{Graph-2.3} demonstrates again underutilized
peripheral clinics clinics(p-value:0.011) despite having similar Booking
Rates. This appears to be more during certain month. Again however due
the small sample sizes it is not possible to perform adequate
statistical analysis.

Initially it seems that the differences although statistically
significant were small. However when consulting the last 4 charts it
seems evident that a notable number of our clinics were underbooked at
80\% booking rate(which translates to 2 clinic slots for 1-man-clinics
and about 3 clinic slots for 2-man-clinics). Although those numbers
warrant attention their statistical significance is not easily
demonstrated due to the small sample sizes(As shown on
\protect\hyperlink{table_1.1}{Table 1.1}). If we were to assume their
significance the next question we need to answer is \emph{why?}.

On further discussion with staff responsible for clinic booking we now
think most deficits in booking rates is due to patient calling in 1 or 2
days before and cancelling. A patient would be considered as DNA only if
he/she was still recorded to come in on the day of the clinic but
didn't.

We suggest one possible option is to over book clinics with 1 -2
patients from the next similar clinic and advising them that this is
only a potential booking and that they will have a garaunteed booking
ontop of that. That they should be ready to be called in 1-2 day's
notice.

This should bring up the booking rate as well as show us a complete true
utilizaition rate to work on next audit cycle. the aim will be to create
a tool whereby this utilization rate is automatically calculated and
used to reasses on a monthly basis.

\begin{center}\rule{0.5\linewidth}{0.5pt}\end{center}

\hypertarget{recommendations}{%
\subsection{4.Recommendations}\label{recommendations}}

We suggest the following recommendations:

\begin{itemize}
\tightlist
\item
  Create waiting list with prebooking patients on multiple clinics
\item
  Allocate Clinic Space by proximity\\
\item
  Monthly Review of Booking Rate\\
\item
  Improve Communication Between Staff accross sites\\
\item
  Disseminate This Report and following monthly Booking Rates to all
  concerned Staff
\end{itemize}

\end{document}
